
This section discusses about the design and architecture of the system

\subsection{Background}

The simulation has been developed as Eclipse Plug-in application which the simulation takes advantage of reusing many existing applications in Eclipse platform. The Eclipse is an extendable platform that provides useful building blocks and frameworks that facilitate developing new seamlessly-integrated applications. These frameworks and building blocks are exposed via well-defined API interfaces, classes, and methods that become mechanisms to use and rules to follow. Plug-in is a smallest unit of Eclipse Platform that can be developed. The Plug-in consist of Java code Java Archive (JAR) library, some read-only files, and other resources such as images, web templates, message catalogs, native code libraries, etc like any normal Java application. However, Plug-in has a manifest that define its connection to other plug-in, extension point, which it is a one of the most important architecture that  makes Eclipse is an extendable platform.

The Eclipse core plug-ins provides many foundation components that required by the simulation such as runtime management, file system, network, data binding, resource management etc. Moreover, there are some external plug-ins that the simulation can make use of beside the core plug-in such as Java Development Tool (JDT), User Interface (UI), and Graphical Editor Framework (GEF) etc.

\subsection{Architecture and Design}

The simulation has been designed with multiple components, in which it helps to reduce the complexity of the whole framework into small components with less complexity. There are four main components that has been divided and shown in the following figure.


\subsubsection{4.1 Loading Environment}

Loading environment is an environment for preparing and setting data that needed for the simulation to execute. The components in the environment are responsible to make sure that every data and information that needs to be executed are created and loaded into the environment before the execution process begins. Also within this environment all data that is needed during runtime must be loaded before the execution process starts. The environment is composed of two components

4.1.1 User Input
This component is an interface between the simulation and user, in which it will extract and validate user algorithm and corresponding topology into the environment. The following will discuss some detail about subcomponents of User Input.
4.1.1.1 Java Editor is a component that allows user to provide an algorithm in Java programming language. The editor provides Java Development Tool (JDT) to help user in coding Java program like any other Java IDE.

4.1.1.2 Graph Editor is a Graphical User Interface (GUI) editor panel that provides a tool for user to specify topology and it properties that the algorithm will run through it as a sample network environment. The editor also allows user to load from or save to a file system as persistence storage in order to reuse a same topology in difference occasion.

4.1.2 Loading Engine
Loading engine is a component that responsible for loading data from User Input and other necessary data into a working space of the simulation before hand over to Runtime Environment. A working space is created by loading engine, in one to one mapping to an execution of the simulation, and used by runtime environment. The working space contains many important data such as topology data, API libraries, and configuration parameters.

4.1.2.1 Topology Data is a data structure that contains all necessary information about the topology created in Graph Editor, in which any change of data by the simulation will affect the display in Graph Editor.

4.1.2.2 API Libraries is a set of all API used by the simulation and user algorithm such as communication API, external plug-in libraries, and adversary program.

4.1.2.3 Configuration Parameters that related to the execution of the simulation configured by user before the start of the simulation such as network environment, number of agents, number of token etc. These configuration parameters are data that must be provided before the starting of the simulation and cannot be modified once the simulation started.

\subsubsection{4.2 Runtime Environment}

Runtime environment is an environment that contains components, in which requires by the simulation during the execution of the simulation such as Execution Engine, Virtual Animation and Adversary Interruption.



4.2.1 Execution Engine

The execution engine is a core engine that drives and manages the execution of the simulation. The execution engine will create an execution process in one to one mapping to a work space that has been created during loading environment. The execution process virtually contains five components that processes user algorithm by using a discrete event processing technique to simulation the execution.




4.2.1.1 Event
An event is a collection of data set, which contains necessary data about the execution and a timestamp to execute this event.

4.2.1.2 Event Queue
An event queue is an ordered queue of events by timestamp. The queue provide a sequence of executed events based on the time that event will be occurred. Any time that a new event arrive the queue will insert the event into an appropriated position based on the timestamp assigned to the event. However, the timestamp will never be smaller than a timestamp of a top event in a queue, in which the time generator will take care of.

4.2.1.3 Time Generator
Timestamp is a time defined by the simulation, called a simulation time unit, the time of the simulation will always be increasing order from the starting of the simulation to the end. A time generator is an engine that will generate a timestamp of each event created by Data Processor. The time generator will assign a new timestamp to a new event based on a current simulation time and condition parameters provided by the data process in which the new time will be equal or bigger than a current.

4.2.1.4 Data Processor
Data processor is an engine that executes an event when a timestamp is equal to a simulation time. Each execution may or may not proceed to a creation of new event or events, which entirely based on the data contains in the event, the simulation configuration parameters and adversary interruption program. Any new event created will be assigned a timestamp of execution and inserted into an event queue. In addition, the execution of event may lead to a change of states of user algorithm and visual animation which discuss in later section.

4.2.1.5 Logger
A logger is a unit that logging events occurred during the execution of the simulation. Information that is logged will be used for replay of the simulation and debugging purposes.


\subsubsection{4.2.2 Topology and Visualization }

The animation of the execution of algorithm will be a simple animation of state and location changes of nodes, agents, boards, tokens and messages. All the display of the animation will be perform in Graph Editor in a basic form of playback and replay.



\subsubsection{4.2.2.1 Playback}
Playback is a series of display changed in Graph Editor while the simulation is executing. The display is changed based on the changes of data stored in work space, and the data in work space can be modified by the execution of event by Execution Engine. Therefore, the simulation use MVC model in managing the data changes and the display of animation.
Replay is a series of display changed in Graph Editor while the simulation is replicating the change of data in work space from a log in file system. The replay engine reads log data and update data in work space in sequences, so, with MVC model the display will update and display as animation.


\subsubsection{.2.3 Adversary Interruption}

The adversary interruption is a sideline program written by user that attached into an engine execution.



The execution engine has data process as an interface to interact with adversary interruption API in order to check and validate the decision based on user implemented adversary program. The user adversary code is mapping one to one to an algorithm and a topology given for an execution.

\subsubsection{.2.3 Probability Libraries Extension}

